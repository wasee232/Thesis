\section{Preliminaries}

Chapter 2
2.1 Preliminaries
2.1.1 Foundations of License Plate Recognition (LPR)
Automatic License Plate Recognition (ALPR) is fundamental  in intelligent transportation systems (ITS), practical uses around us such as traffic surveillance, law and order, toll gathering, and smart parking management(1) . The ALPR pipeline usually consists of four main phrases: image learning, license plate localization, character segmentation, and optical character recognition (OCR)(1). While traditional ALPR systems run on hand-crafted features and heuristic-based computer vision techniques, the field has shifted towards deep learning, which offers superior and adaptation to real-world variability (2).
The performance of ALPR systems is highly sensitive to environmental factors for instance, images often suffer from motion blurness (due to vehicle or camera’s motionlessness), low-light factors (night-time, shadows), different weather type (rain, fog, snow, storm), and different plate designs (fonts, scripts) (1). These challenges are particularly some-what common for non-Latin scripts such as Bangla, which feature complex structure and different words(symbols) (3). Addressing these basic problems requires advanced models and preprocessing pipelines capable of enhancement of  image quality and extracting core features under challenging conditions mentioned above.
2.1.2 Deep Learning Models for LPR
Convolutional Neural Networks (CNNs) have become unavoidable of current LPR systems due to their ability to learn hierarchical features from raw images provided (4). CNNs is capable of capturing both low-level (edges, textures) and high-level (shapes, objects) patterns in image, making them best for both license plate detection and character identification. Architectures such as VGG, ResNet, and physically coded lightweight CNNs have been used by losing accuracy and computational cost.
The YOLO (You Only Look Once) has a different version which has made changes in real-time object detection, including license plate identification (7). YOLO’s single-stage identification architecture enables high-speed inference, making it best/ ideal for real-time applications for ALPR. Sequential versions have introduced improvements for example, offer optimized accuracy-speed trade-offs and improved small object detection, which is critical for recognizing license plates in unconstrained environments.
Generative Adversarial Networks (GANs) have been introduced as powerful tools for image restoration, including deblurring, super-resolution, and data augmentation (8). GANs has  a generator and  adversarial training, enabling the making of realistic images from degraded inputs. In LPR, GANs are used to restore motion-blurred or low-resolution license plates, generate synthetic training data, and enhance image quality prior to recognition (8,9). Variants such as ESRGAN, CycleGAN, and LPSRGAN have  restored fine details and improved downstream OCR accuracy.(8,5)
Sequence modeling is necessary for recognizing the order of characters on a license plate. LPRNet is a lightweight, segmentation-free CNN architecture that identifies character sequences directly from license plates , using CTC (Connectionist Temporal Classification) loss for alignment (4,10). CRNN (Convolutional Recurrent Neural Network) combines CNNs for feature extraction with RNNs such as LSTM, GRU for sequence modeling, to ensure robust identification of variable-length strings without explicit segmentation(2,5). Recently, transformer-based models have been discovered for both detection and recognition, leveraging up for self-attention mechanisms to get long-range dependencies and global context, which is particularly helpful for complex scripts and distorted images (5,7).
2.1.3 Optical Character Recognition (OCR) for Bangla and Non-Latin Scripts
The process of converting an image to machine code is called OCR. While   for Latin scripts, OCR for Bangla presents unique challenges due to its complex nature(curvy), compound characters  (3). Traditional OCR frames such as Tesseract and EasyOCR have adopted Bangla text but usually struggle with low-quality images, complex layouts, and script-specific features. Hybrid model combining YOLO for character detection and advanced CNNs (e.g., EfficientNet, ResNet) for identification have achieved state-of-the-art results in Bangla OCR.(3)


\section{Review of Existing Research}
Perform a literature survey and find relevant materials and information.

\section{Summary of Key Findings}

Discuss key results, patterns or trends, and implications.