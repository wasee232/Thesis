%\section{Introduction}
\section{Background} 
Review of literature and background study \\

\textbf{[In case of internship - Company Background]}
Describe the company where you completed your internship. Include information such as the company’s history, mission, vision, key products or services, and its position in the industry. Highlight any relevant details that provide context for the project or tasks you worked on.

\section{Rational of the Study or Motivation}
This section discusses the significance and relevance of the research gap and its impact.

\section{Problem Statement}
Nowadays, Automatic license-plate recognition (ALPR) systems are built on law enforcement, smart transportation and traffic-safety surveillance infrastructures.However, due to movement of vehicles at high speed, the amount of motion blur and geometric distortion is induced into the images and there is also a significant difficulty to address non-Latin scripts, such as Bangla compared to standard english digits where the structural complexity of the numerals and the contacts between the characters (Matra) place a greater thinking burden on the observer[1]. Even though few researchers have addressed a solution, there remains some technical gaps.Even the previous studies focus on restoration of images using GFPGAN which specially targets low-resolution motion blur but not the high frequency motion blur[1].Moveover,Other studies by  Wang et al. [2], uses YOLOv5 to ensure environment interference and lilted plates are restored which do not include a fast moving car’s licence plate distortion fixing Where as 
low-light enhancement studies that use URetinex-Net [3] and U-Net or YOLOv10 [4], [5]
Handles atmospheric noises illumination changes fails when there is motion-induced blur for more demanding character level integrity. So to overcome this gap our study will propose a  powerful multi-stage deep learning architecture which will handle the motion-blurred and geometrically distorted Bengali license plates.Our suggested design uses the YOLOv8 to identify the object accurately, DeblurGANv2 to evaluate the high-speed image enhancement specifics, and a Bangla-trained OCR model to consider the script-specifics.This system offers a real world reliability which involves heavy traffic and environments aimed at preventing crime,thus providing for Bangladesh's intelligent transportation infrastructure an adaptable approach.


\section{Objective}
Our goal for this research is to design a multi-stage, multifaceted deep learning system which is designed to detect and recognize Bangla license plates in extreme motion blur and geometric distortion in images. This study deals with the significant decrease in performance of license plate detection for a fast moving vehicle in uncontrolled condition. For complex urban backgrounds our research uses state-of-the-art YOLOv8 architecture to make sure the detection is highly sensitive. One of the important stages of our study is to use DeblurGANv2 to reduce the loss of character-level accuracy, which is a denoising assistant to restore high-frequency texture detail in the image of a license plate when in motion. Additionally, to address the unique problems of the non-Latin character structure and matra relations that cannot be solved by the traditional models, our study intends to give a specialized optical character recognition (OCR) system which will be trained on complex Bengali script. Therefore, the ultimate goal of this study is to build an effective pipeline which significantly increases reliability and recognition accuracy in real world systems such as highway surveillance and criminal investigations. In order to improve the effectiveness of law enforcement operations and  intelligent transportation systems in Bangladesh, our research may offer an effective In order to improve intelligent transportation systems and the effectiveness of law enforcement operations in Bangladesh, this study may offer a scalable and effective way to address the academic gap between current fixing techniques and the unique script requirements of this nation.

\section{Methodology in Brief }

This section provides a concise overview of how the study was conducted. State the research approach, data collection, analysis, etc.


\section{Scopes and Challenges}
Briefly outline the boundaries of your study and any constraints.

\section{Team Overview}
\textbf{[In case of internship]}

Write about your group and your role in that team.

\section{Key Learnings and Insights}
\textbf{[In case of internship]}

This section can be used to introduce the team members involved in the internship project, highlighting their roles, contributions, and any collaborative efforts. It provides context on the team dynamics and how each member contributed to the overall success of the project.




